\documentclass[a4paper, 14pt]{report}

%% PACKAGES %%%%%%%%%%%%%%%%%%%%%%%%%%%%%%%%%%%%%%%%%%%%%%%%

\usepackage{amsmath}  % math
\usepackage{graphicx} % embedding images
\usepackage{enumitem} % lists
\usepackage{hyperref} % urls
\usepackage{logicproof} % fitch style proofs

\usepackage{geometry} % margins
\geometry{margin=0.9in} % make the margin smaller

\graphicspath{{z:/Documents/Assessments/Logic/img}{../img/}{./img/}}

%% END OF PACKAGES %%%%%%%%%%%%%%%%%%%%%%%%%%%%%%%%%%%%%%%%%%

\author{%
	Norbert Logiewa \\
	\small{nl253}}

\date{November 2017}

\title{%
	Assessment 1 \\
	\Huge{%
		Regular Languages \\
		and \\
		Logic}}

% MACROS %%%%%%%%%%%%%%%%%%%%%%%%%%%%%%%%%%%%%%%%%%%%%%%%%%%%%

\newcommand{\topic}[1]{%
	\pagebreak
	\section*{%
		\begin{center} \huge{#1} \end{center}}}

\newcommand{\centeredimg}[1]{%
	\begin{figure}[h]
		\begin{center}
			\includegraphics[height=8cm]{#1}
		\end{center}
\end{figure}} 

\newcommand{\answer}[1]{%
	\begin{flushleft}
		\textbf{Answer}:\\
			#1
	\end{flushleft}}

\newcommand{\question}[1]{\subsection*{#1}}
\newcommand{\task}[1]{%
	\begin{flushleft}
		\textbf{Task:}\\ 
		#1
	\end{flushleft}}

% END OF MACROS %%%%%%%%%%%%%%%%%%%%%%%%%%%%%%%%%%%%%%%%%%%%%%%

\begin{document}

\maketitle

\topic{Propositional Logic}

\question{1. Model as a propositional formal the property that
	the initial state cannot be reached from $c$ or $b$}

\answer{%
	$ model \rightarrow \neg ((c \rightarrow a^\prime) \vee (b \rightarrow a^\prime)) $
}

\question{2. Prove the following sequent:}

\[ a \rightarrow b^\prime.b \rightarrow c^\prime.c \rightarrow b^\prime \vdash (a \vee b) \rightarrow (b^\prime \vee c^\prime)\]

\hline

{\setlength\subproofhorizspace{4em}
	\begin{logicproof}{1}
		a \rightarrow b^\prime.b \rightarrow c^\prime.c \rightarrow b^\prime & premise \\
		(a \rightarrow b^\prime) \wedge (b \rightarrow c^\prime) \wedge (c \rightarrow b^\prime) & $\wedge_i$\ 1 \\
		(a \rightarrow b^\prime) \wedge (c \rightarrow b^\prime) \wedge (b \rightarrow c^\prime) & commutativity of $\wedge$\ 2 \\
		(a \wedge c \rightarrow b^\prime) \wedge (b \rightarrow c^\prime) & distributivity of $\rightarrow$ 3 \\
		a \wedge c \rightarrow b^\prime & $\wedge_{e1}$ 4 \\
		a \rightarrow b^\prime & $\wedge_{e1}$ 5 \\
		a \vee b \rightarrow b^\prime & $\vee_{i2}$ 6 \\
		(a \vee b) \rightarrow (b^\prime \vee c^\prime) & $\vee_{i2}$ 7
	\end{logicproof}}

\question{3. Model the property that the automate can only be in one state at a time via a definition of a propositional function}
\answer{%
	$ A\oplus B \equiv (A \vee B) \wedge (\neg A \vee \neg B) $ [definiton of the $ XOR $ operator]\\ \\
	$ goodState(a, b, c) \equiv (a \rightarrow b^\prime) \oplus (b \rightarrow c^\prime) \oplus (c \rightarrow b^\prime) $ 
}

\question{4. Negate the formula $ ((a \rightarrow b^\prime) \wedge (a \wedge \neg b) \wedge (b^\prime \wedge \neg a^\prime)) \rightarrow (b \rightarrow \neg a^\prime) $  and convert it to CNF.\\ }

\begin{enumerate}

	\begin{enumerate}

		\item Negate the formula
			$ \neg (((a \rightarrow b^\prime) \wedge (a \wedge \neg b) \wedge (b^\prime \wedge \neg a^\prime)) \rightarrow (b \rightarrow \neg a^\prime)) $
		
		\item Remove $\rightarrow$ using $A \rightarrow B \equiv \neg A \vee B$ property

			\begin{enumerate}		
				\item $ \neg (((\neg a \vee b^\prime) \wedge (\neg a \vee \neg b) \wedge (b^\prime \wedge \neg a^\prime)) \rightarrow (\neg b \vee \neg a^\prime)) $ 
				\item $ \neg (\neg ((\neg a \vee b^\prime) \wedge (\neg a \vee \neg b) \wedge (b^\prime \wedge \neg a^\prime)) \vee (\neg b \vee \neg a^\prime)) $ 
			\end{enumerate}		

		\item Push $\neg$ inwards using De Morgan's Laws
		
			\begin{enumerate}		
				\item $ ((\neg a \vee b^\prime) \wedge (\neg a \vee \neg b) \wedge (b^\prime \wedge \neg a^\prime)) \wedge \neg (\neg b \vee \neg a^\prime) $
				\item $ ((\neg a \vee b^\prime) \wedge (\neg a \vee \neg b) \wedge (b^\prime \wedge \neg a^\prime)) \wedge (b \wedge a^\prime) $
			\end{enumerate}		

		\item Distribute

			\begin{enumerate}		
				\item $ b \wedge ((\neg a \vee b^\prime) \wedge (\neg a \vee \neg b) \wedge (b^\prime \wedge \neg a^\prime)) \wedge a^\prime \wedge  ((\neg a \vee b^\prime) \wedge (\neg a \vee \neg b) \wedge (b^\prime \wedge \neg a^\prime)) $ 
				\item $ b \wedge (\neg a \vee b^\prime) \wedge b \wedge (\neg a \vee \neg b) \wedge b \wedge (b^\prime \wedge \neg a^\prime) \wedge a^\prime \wedge (\neg a \vee b^\prime) \wedge a^\prime \wedge (\neg a \vee \neg b) \wedge a^\prime \wedge (b^\prime \wedge \neg a^\prime) $
			\end{enumerate}		

			\pagebreak

		\item Simplify (remove duplicate elements because $ A \wedge A \equiv A$)
		
			\begin{enumerate}		
				\item $ (\neg a \vee b^\prime) \wedge b \wedge (b^\prime \wedge \neg a^\prime) \wedge (\neg a \vee \neg b) \wedge a^\prime  $ \\ 
				\item $ (\neg a \vee b^\prime) \wedge b \wedge b^\prime \wedge \neg a^\prime \wedge (\neg a \vee \neg b) \wedge a^\prime  $ \\  \\
					Parenthesis are dropped because: $ A \wedge B \wedge C \equiv A \wedge (B \wedge C) \equiv (A \wedge B) \wedge C $
					so they all need to be $True$ anyway \\
				\item $ \neg a^\prime \wedge a^\prime \wedge b \wedge b^\prime \wedge (\neg a \vee \neg b)  \wedge (\neg a \vee b^\prime) $ \\ 
			\end{enumerate}		

	\end{enumerate}		
\end{enumerate}		

\hline

\answer{$ \neg a^\prime \wedge a^\prime \wedge b \wedge b^\prime \wedge (\neg a \vee \neg b)  \wedge (\neg a \vee b^\prime) $ \\}

\question{5. Explain the principle of unit propagation from DPLL, and use
  it to show that CNF formula of the previous question is unsatisfiable.}

\begin{enumerate}		

	\item \answer{%
			Unit propagation refers to a technique that is used to simplify
			logical formulas.
			When we have a clause that \textit{only} contains a single literal
			such as $ x $, in all other clauses that contain $ x $, we can
			replace it with $ True $ ie. $\top$.\\ }


	\item \answer{We can use it to show that the previous formula is
		unsatisfiable by demonstrating that it is a contradiction.
		We simplify it and continue to substitute $True$
		ie $\top$ for single literal clauses.

		\begin{enumerate}		
			\item $ \neg a^\prime \wedge a^\prime \wedge b \wedge b^\prime \wedge (\neg a \vee \neg b)  \wedge (\neg a \vee b^\prime) $ 
			\item $ \neg a^\prime \wedge a^\prime \wedge b \wedge \top \wedge (\neg a \vee \neg b)  \wedge (\neg a \vee \top) $ replace all $b^\prime$s with $\top$  \\
			\item $ \neg a^\prime \wedge a^\prime \wedge \top \wedge \top \wedge (\neg a \vee \bot)  \wedge (\neg a \vee \top) $ replace all $b$s with $\top$  \\
			\item $ \bot \wedge \top \wedge \top \wedge \top \wedge (\neg a \vee \bot)  \wedge (\neg a \vee \top) $ replace all $a^\prime$s with $\top$  \\
		\end{enumerate}			

		When we rearrange and simplify the formula, it becomes clear that we have 
	  $\top \wege \bot$, this is a contradiction, something cannot be $True$ and $False$ at the same time ($\bot \wedge \top \equiv \bot$).}

\end{enumerate}		

\question{6. Comment on the validity of the original formula.} 

\answer{The formula is not satisfiable and therefore not valid as satisfiability is
	a prerequisite for validity. In other words, because that logical statement
	was a contradiction (\textit{assumed something can be False $\bot$ and True $\top$ at the same time}), there was no way in which it could have been true and
	because of that it is not valid.}

\topic{First Order Logic}

\question{1. Write a formula in first-order logic, using the parent relation, 
	that states that two entities $x$ and $y$ are siblings if they share a parent} 

\answer{$ \forall{i}.\forall{j}.siblings(i, j) \equiv (parent(x, i) = parent(x, j)) $}

\question{2. Assuming there is an equality relation $ \equiv $ on $ P $ 
	such that $ x \equiv y $ means $ x $ and $ y $ are the same entity, 
  write a formula in first order logic stating that every entity has two distinct parents.} 

\answer{%
	$ \exists!{x}.P(x) \equiv \exists{x}(P(x) \wedge \forall{y}(P(y) \rightarrow y = x))  $
	\footnote{There exists exactly one, see https://math.stackexchange.com/questions/228285/how-can-i-get-the-negation-of-exists-unique-existential-quantification} \\
	$ \forall{i}.\exists!{y}.\exists!{x} (parent(x, i) \wedge parent(y, i) \wedge \neg (x \equiv y)) $ 
}

\question{3. Prove that the following sequent is valid:}

\[ parent(p, q), \forall{x}.\forall{y}(parent(x, y) \rightarrow \exists{z}.parent(z,x)) \vdash \exists{z}.parent(z, p) \\ \]

\hline

{\setlength\subproofhorizspace{2em}
	\begin{logicproof}{1}
		parent(p, q) & premise \\
		\forall{x}.\forall{y}(parent(x, y) \rightarrow \exists{z}.parent(z,x)) & premise \\ 
		\begin{subproof}
			parent(p, q) \rightarrow \exists{z}.parent(z, p) & \forall e\ 2 \\
			parent(p, q) & 1 \\
			\exists{z}.parent(z, p) & modus ponens
		\end{subproof}
		\exists.parent(z, p) & conclusion
	\end{logicproof}}

\question{4. Interpret the following formula into simple English statement}

$ \forall{x}.\exists{y}.parent(y, x) \rightarrow \forall{a}.\exists{b}.parent(a, b) $

\answer{If all children have a parent, then all parents have a child.}

\question{5. The sequent below is not valid. Show this by proving a 
	counter-example: a concrete definition of the universe $ P $ (a set) 
	and the relation parent. Explain why this is a counterexample.} 

\[ 
	\vdash \forall{x}.\exists{y}.parent(y, x) \rightarrow \forall{a}.\exists{b}.parent(a, b) 
\]

\answer{If this is not valid then:
	$ \forall{x}.\exists{y}.parent(y, x) \wedge \neg (\forall{a}.\exists{b}.parent(a, b)) $,
	because the only way implication can be wrong is if we are concluding
	$False$ from $True$ ie $ \neg (A \rightarrow B) \equiv A \wedge \neg B$.
	So I need to prove this is true by defining the Universe $P$ (a set):

	\[ P = \{\ x\ |\ \exists{y}.parent(y, x)\} \]

	and the case that contradicts it:

	\[ \exists{a}.\exists{b}.\neg parent(a, b) \]

	The reason this would be a counterexample is that it shows that
	there is an object for which the predicate $ parent $ does not hold,
	which contradicts this logical statement which is universally
	quantified ie.\ makes a claim about all objects.}

\pagebreak

\question{6. What is the smallest possible counterexample universe $ P $ and relation parent. Explain your reasoning.}

\answer{The smallest possible counterexample universe (a set) would consist
	of the minimum number of elements for which the predicate does not hold (here this is two). 
	Normally one counterexample  is sufficient to contradict any
  statement that is universally quantified. In other words $\exists(A).\neg P(A)$ 
  contradicts $\forall{A}.P(A)$, but because this predicate is a statement
  about two objects $A$ and $B$ we do need both to show a single case where
  $P$ does not hold.}

\question{7. Prove that the following sequent is valid.}

\[ \forall{x}.\forall{y}.parent(x, y) \rightarrow \neg(x \equiv y) \vdash \neg \exists{x}.parent(x, x) \]

\hline

% \begin{enumerate}		
	% \item $ \forall{x}.\forall{y}.parent(x, y) \rightarrow \neg(x \equiv y) $ [premise] \\
	% \item $ parent(x_0, y_0) \rightarrow \neg (x_0 \equiv y_0)  $ $ \forall_e $ 1
	% \item $ \neg parent(x_0, y_0) \vee \neg (x_0 \equiv y_0) $  because $ P \rightarrow Q \equiv \neg P \vee Q $
  % \item $ \neg (parent(x_0, y_0) \wedge (x_0 \equiv y_0)) $  De Morgan's Law 
	% \item $ \neg parent(x_0, x_0) $  because if $ (A \equiv B ) $ then we can replace all $B$ with $A$ or $A$ with $B$ 
	% \item $ \neg \exists{x_0}.parent(x_0, x_0)) $  $ \exists_i $ \\
	% \hline
	% \item $ \neg \exists{x_0}.parent(x_0, x_0)) $  $ \exists_i $
% \end{enumerate}		

{\setlength\subproofhorizspace{0em}
	\begin{logicproof}{1}
		\forall{x}.\forall{y}.parent(x, y) \rightarrow \neg(x \equiv y) & premise \\
		\begin{subproof}
			parent(x_0, y_0) \rightarrow \neg (x_0 \equiv y_0)  & $ \forall_e $ 1 \\
			\neg parent(x_0, y_0) \vee \neg (x_0 \equiv y_0) &  because $ P \rightarrow Q \equiv \neg P \vee Q $ 2 \\
			\neg (parent(x_0, y_0) \wedge (x_0 \equiv y_0)) &  De Morgan's Law 3 \\
			\neg parent(x_0, x_0) & $\wedge_{e} $ 4 \\
			\neg \exists{x_0}.parent(x_0, x_0) &  $ \exists_i $ 5
		\end{subproof}
		\neg \exists{x}.parent(x, x) & conclusion
	\end{logicproof}}

\topic{Regular Lanugages and Finite Automata}

\question{1. For the reular expression $ (A|AB|ABB)^* $ do the following steps \\(and justify them):}

\begin{itemize}		
	\item translate it into an NFA \\
		\answer{\centeredimg{rl_1_a.jpg}}
	\item remove (semantic preserving) the $ \epsilon $ transitions \\
		\answer{--} 
	\pagebreak
	\item create the corresponding DFA \\
		\answer{\centeredimg{rl_1_c.jpg}}
\end{itemize}

\question{2. Consider the following language: A word is in the language if
  and only if it contains an even number of $A$s an odd number of $B$s and precisely one $C$.}

\answer{This is not possible because Finite Automata \textit{cannot store any states}
	-- they don't have memory and cannot keep track of how many $A$s
	and $B$s have been encountered. They only memory they have is the state they are in.
	Generally speaking FSM lack the expressivenes to solve this problem
	and parsing needs to be used in such cases.} 

\pagebreak

\section*{References}

\begin{flushleft}
	\begin{itemize}[noitemsep]
		\item \url{https://en.wikipedia.org/wiki/Conjunctive_normal_form}
		\item \url{https://en.wikipedia.org/wiki/De_Morgan\%27s_laws}
		\item \url{https://en.wikipedia.org/wiki/Distributive_property}
		\item \url{https://en.wikipedia.org/wiki/Exclusive_or}
		\item \url{https://en.wikipedia.org/wiki/Finite-state_machine}
		\item \url{https://en.wikipedia.org/wiki/Modus_ponens}
		\item \url{https://en.wikipedia.org/wiki/Uniqueness_quantification}
		\item \url{https://math.stackexchange.com/questions/228285/how-can-i-get-the-negation-of-exists-unique-existential-quantification}
		\item \url{https://stackoverflow.com/questions/133601/can-regular-expressions-be-used-to-match-nested-patterns}
		\item \url{https://en.wikipedia.org/wiki/Negation}
	\end{itemize}		
\end{flushleft}

\end{document}
%% vim: foldmethod=marker conceallevel=0:
