\documentclass[a4paper, 13pt, draft]{report}
\usepackage{amsmath}

\author{%
	Norbert Logiewa \\
\small{nl253}}

\title{%
	\Huge{Assessment 1} \\
	\small{Logic and Regular Languages}
}

\begin{document}

\maketitle

\newcommand{\answer}[1]{%
\begin{flushleft}
		
	\textbf{Answer}:\\
		#1
\end{flushleft}
}

\section*{\begin{center} \huge{Propositional Logic} \end{center}} 

	\begin{center} 
		$ model = (a \rightarrow b^\prime) \wedge (b \rightarrow c^\prime) \wedge (c \rightarrow b^\prime) $ 
	\end{center}

	\subsection*{%
		1. Model as a propositional formal the poroperty that 
	the inital state cannot be reached from c or b}

	\[ model \rightarrow \neg ((c \rightarrow a^\prime) \vee (b \rightarrow a^\prime)) \]

	\subsection*{2. Prove the following sequent:} %% TODO

	\[ a \rightarrow b^\prime.b \rightarrow c^\prime.c \rightarrow b^\prime \vdash (a \vee b) \rightarrow (b^\prime \vee c^\prime)\]

	\begin{enumerate}		
		\item $ a \rightarrow b^\prime.b \rightarrow c^\prime.c \rightarrow b^\prime $ [premise]
		\item $ (a \rightarrow b^\prime) \wedge (b \rightarrow c^\prime) \wedge (c \rightarrow b^\prime) $ $ \wedge_i $
		\item $ (a \rightarrow b^\prime) \wedge (c \rightarrow b^\prime) \wedge (b \rightarrow c^\prime) $ commutativity of $ \wedge $
		\item $ ((a \wedge c) \rightarrow b^\prime) \wedge (b \rightarrow c^\prime) $ distributivity of $ \rightarrow $
		% \item $ \neg (a \wedge c) \vee b^\prime \wedge (\neg b \vee c^\prime) $ De Morgan's Law
		% \item $ \neg a \vee c \vee b^\prime \vee b \wedge \neg c^\prime $
		% \item $ \neg a \vee c \vee b^\prime \vee b \wedge \neg c^\prime $
		% \item $ (\neg a \vee b^\prime) \wedge (\neg b \vee c^\prime) \wedge (\neg c \vee b^\prime)$  De Morgan's Law
		% \item $ (\neg a \vee b^\prime) \wedge (\neg c \vee b^\prime) \wedge (\neg b \vee c^\prime) $  commutativity of $ \wedge $
		% \item $ (b^\prime \vee \neg a) \wedge (b^\prime \vee \neg c) \wedge (\neg b \vee c^\prime) $  commutativity of $ \vee $
		% \item $ b^\prime \vee ( \neg a \wedge \neg c) \wedge (\neg b \vee c^\prime) $  distributivity 
		% \item $ b^\prime \vee ( \neg a \wedge \neg c) \wedge (\neg b \vee c^\prime) $ 
		% \item $(a \vee b) \rightarrow (b^\prime \vee c^\prime)$  [conclusion] 1 -- ?
	\end{enumerate}		

	\subsection*{3. Model the property that the automate can only be in one stat at a time via a defintion of a propositional function}

	\begin{center}
		$ goodState(a, b, c) \equiv (a \rightarrow b^\prime) \oplus (b \rightarrow c^\prime) \oplus (c \rightarrow b^\prime) $
	\end{center}

	\subsection*{4. Negate the formula and convert it to CNF} %% TODO check !!!

	\[ ((a \rightarrow b^\prime) \wedge (a \wedge \neg b) \wedge (b^\prime \wedge \neg a^\prime)) \rightarrow (b \rightarrow \neg a^\prime) \]

	\hline 
	  
	\begin{enumerate}		
		\item  $ \neg (((a \rightarrow b^\prime) \wedge (a \wedge \neg b) \wedge (b^\prime \wedge \neg a^\prime)) \rightarrow (b \rightarrow \neg a^\prime)) $
		\item  $ \neg (((\neg a \vee b^\prime) \wedge (a \wedge \neg b) \wedge (b^\prime \wedge \neg a^\prime)) \rightarrow (\neg b \vee a^\prime)) $
		\item  $ \neg (\neg ((\neg a \vee b^\prime) \wedge (a \wedge \neg b) \wedge (b^\prime \wedge \neg a^\prime)) \vee (\neg b \vee a^\prime)) $
		\item  $ \neg ((\neg (a \vee b^\prime) \vee \neg (a \wedge \neg b) \vee \neg (b^\prime \wedge \neg a^\prime)) \vee (\neg b \vee a^\prime)) $ 
		\item  $ \neg (\neg (a \vee b^\prime) \vee \neg (a \wedge \neg b) \vee \neg (b^\prime \wedge \neg a^\prime)) \wedge \neg (\neg b \vee a^\prime) $ 
		\item  $ (a \vee b^\prime) \wedge (a \wedge \neg b) \wedge (b^\prime \wedge \neg a^\prime) \wedge \neg (\neg b \vee a^\prime) $ 
		\item  $ (a \vee b^\prime) \wedge \neg (\neg a \vee b) \wedge \neg (\neg b^\prime \vee a^\prime) \wedge \neg (\neg b \vee a^\prime) $
		\item  $ (a \vee b^\prime) \wedge \neg (\neg a \vee b) \wedge \neg (\neg b^\prime \vee a^\prime) \wedge \neg (\neg b \vee a^\prime) $ commutativity
		\item $ (a \vee b^\prime) \wedge \neg (\neg a \vee b) \wedge \neg (\neg b^\prime \vee a^\prime) \wedge \neg (\neg b \vee a^\prime) $
		\item $ (a \vee b^\prime) \wedge a \wedge \neg b \wedge \neg (\neg b^\prime \vee a^\prime) \wedge \neg (\neg b \vee a^\prime) $
		\item $ (a \vee b^\prime) \wedge a \wedge \neg b \wedge b^\prime \wedge \neg a^\prime \wedge \neg (\neg b \vee a^\prime) $
		\item $ (a \vee b^\prime) \wedge a \wedge \neg b \wedge b^\prime \wedge \neg a^\prime \wedge b \wedge \neg a^\prime $
	\end{enumerate}	
	\hline
	\[ (a \vee b^\prime) \wedge a \wedge \neg b \wedge b^\prime \wedge \neg a^\prime \wedge b \wedge \neg a^\prime \]

\subsection*{%
	5. Explain the principle of unit propagation from DPLL, 
	and use it to show that CNF formula of the previous question is unsatisfiable.}

	Unit propagation refers to a technique that is used to simplify
	logical formulas.
	When we have a clause that \textit{only} constains a single literal
	such as $ x $, in all other clauses that contain $ x $, we can
	subsitute it for $ True $.

	We can use it to show that the previous formula is unsatisfiable by demonstrating that it is a contradiction.
	If we continue to substitute $ True $ for single literal clauses:


	\[ 
		(True \vee b^\prime) \wedge True \wedge \neg b \wedge b^\prime \wedge \neg a^\prime \wedge b \wedge \neg a^\prime 
	\]

	\answer{Here, if we rearrange the formula, it becomes clear that we have both $ b $ and $ \neg b $, this is a contradiction.}

	\[ 
		(True \vee b^\prime) \wedge True \wedge b \wedge \neg b \wedge b^\prime \wedge \neg a^\prime
	\]

\subsection*{6. Comment on the validity of the original formula.} 

\section*{\begin{center}\huge{First Order Logic}\end{center}} 

	\subsection*{%
		1. Write a formula in first-order logic, using the parent relation, 
	that states that two entities $x$ and $y$ are siblings if they share a parent}

	\[ \forall{i}.\forall{j}.siblings(i_0, j_0) \equiv (parent(x, i_0) = parent(x, j_0)) \]

	\subsection*{%
		2. Assuming there is an equality relation $ \equiv $ on $ P $ 
		such that $ x \equiv y $ means $ x $ and $ y $ are the same entity, 
	write a formula in first order logic stating that every entity has two distinct parents.}

	\[ \forall{i}.\exists!{y}.\exists!{x} (parent(x, i) \wedge parent(y, i) \wedge \neg (x \equiv y)) \]

	\subsection*{3. Prove that the following sequent is valid:}

	\[ parent(p, q).\forall{x}.\forall{y}(parent(x, y) \rightarrow \exists{z}.parent(z,x)) \vdash \exists.parent(z, p) \]

	\hline

	\begin{enumerate}		
		\item $ parent(p, q).\forall{x}.\forall{y}(parent(x, y) \rightarrow \exists{z}.parent(z,x)) $ [premise]
		\item  ---
		\item $ \exists.parent(z, p) $ [conclusion] 1 -- ?
	\end{enumerate}		

	\hline

\subsection*{4. Interpret the following formula into simple English statement}

\[ 
	\forall{x}.\exists{y}.parent(y, x) \rightarrow \forall{a}.\exists{b}.parent(a, b)
\]

\answer{If all children have a parent, then all parents have a child.}

\subsection*{5. The sequent below is not valid. Show this by proving a 
	counter-example: a concrete definition of the universe $ P $ (a set) 
	and the relation parent. Explain why this is a counterexample.}

\[ 
	\vdash \forall{x}.\exists{y}.parent(y, x) \rightarrow \forall{a}.\exists{b}.parent(a, b) 
\]

\begin{flushleft}
	\textbf{In English}\\
	if all children have a parent then all parents have a child.
\end{flushleft}

\begin{enumerate}		
	\item $ \forall{x}.\exists{y}.parent(y, x) \rightarrow \forall{a}.\exists{b}.parent(a, b) $ [premise] \\ 
  \hline
	\item  $ \let{P} = \{\ x\ |\ x\ \} $  \\
  \hline
	\item $ False $ [conclusion]
\end{enumerate}		


\subsection*{6. What is the smallest possible counterexample universe $ P $ and relation parent. Explain your reasoning.}

\subsection*{7. Prove that the following sequent is valid.}

\[
	\forall{x}.\forall{y}.parent(x, y) \rightarrow \neg(x \equiv y) \vdash \neg \exists{x}.parent(x, x)
\]

ie. if someone is a parent and has a child, then that child cannot be it's parent.

\begin{enumerate}		
	\item $ \forall{x}.\forall{y}.parent(x, y) \rightarrow \neg(x \equiv y) $ [premise] \\
	\hline
	\item $ parent(x_i, y_j) \rightarrow \neg (x_i \equiv y_j)  $ $ \forall_e $ 1
	\item $ \neg parent(x_i, y_j) \vee \neg (x_i \equiv y_j) $  because $ P \rightarrow Q \equiv \neg P \vee Q $
	\item $ \neg (parent(x_i, y_j) \wedge (x_i \equiv y_j)) $  De Morgan's Law \\
	\hline
	\item $ \neg \exists{x}.parent(x, x) [conclusion] $
\end{enumerate}		



\end{document}

	%% vim: foldmethod=marker conceallevel=0:
