\documentclass[a4paper, 14pt]{report}

\usepackage{amsmath}  % math
\usepackage{graphicx} % embedding images
\usepackage{enumitem}
\usepackage{geometry} % margins
\geometry{margin=1in} % make the margin smaller

\graphicspath{ {/home/norbert/Documents/Assignments/Logic/img/} }

\author{%
	Norbert Logiewa \\
\small{nl253}}

\date{November 2017}

\title{%
	Assessment 1 \\
	\Huge{%
		Logic, Regular Languages,\\  
		and \\
		Finite Automata}}

% MACROS %%%%%%%%%%%%%%%%%%%%%%%%%%%%%%%%%%%%%%%%%%%%%%%%%%%%%

\newcommand{\topic}[1]{%
	\pagebreak
  \section*{\begin{center} \Huge{#1} \end{center}}
}

\newcommand{\centeredimg}[1]{%
		\begin{figure}[h]
			\begin{center}
				\includegraphics[height=5cm]{#1}
			\end{center}
		\end{figure}
} 

\newcommand{\answer}[1]{%
	\begin{flushleft}
		\textbf{Answer}:\\
			#1
	\end{flushleft}
}

\newcommand{\question}[1]{%
	\section*{%
		#1 
	}
}

% END OF MACROS %%%%%%%%%%%%%%%%%%%%%%%%%%%%%%%%%%%%%%%%%%%%%%%

\begin{document}

\maketitle


\topic{Propositional Logic}

	\question{%
		1. Model as a propositional formal the poroperty that 
	the inital state cannot be reached from c or b}

	\answer{%
		$ model \rightarrow \neg ((c \rightarrow a^\prime) \vee (b \rightarrow a^\prime)) $
	}

	\question{2. Prove the following sequent:} %% TODO

	\[ a \rightarrow b^\prime.b \rightarrow c^\prime.c \rightarrow b^\prime \vdash (a \vee b) \rightarrow (b^\prime \vee c^\prime)\]

	\begin{enumerate}		
		\item $ a \rightarrow b^\prime.b \rightarrow c^\prime.c \rightarrow b^\prime $ [premise] \\
		\hline
		\item $ (a \rightarrow b^\prime) \wedge (b \rightarrow c^\prime) \wedge (c \rightarrow b^\prime) $ [$ \wedge_i $] 1
		\item $ (a \rightarrow b^\prime) \wedge (c \rightarrow b^\prime) \wedge (b \rightarrow c^\prime) $ [commutativity of $ \wedge $] 2
		\item $ (a \wedge c \rightarrow b^\prime) \wedge (b \rightarrow c^\prime) $ [distributivity of $ \rightarrow $] 3
		\item $ a \wedge c \rightarrow b^\prime $ [$ \wedge_{e2} $] 4
		\item $ a \rightarrow b^\prime $ [$ \wedge_{e2} $] 5
		\item $ a \vee b \rightarrow b^\prime $ [$ \vee_{i2} $] 6 \\
		\hline
		\item $ (a \vee b) \rightarrow (b^\prime \vee c^\prime) $  [$ \vee_i $] 7 [conclusion] \\
	\end{enumerate}		

	\question{3. Model the property that the automate can only be in one stat at a time via a defintion of a propositional function}

	\answer{%
		$ A \oplus B \equiv (A \vee B) \wedge (\neg A \vee \neg B) $ [definiton of the $ XOR $ operator] \\
			$ goodState(a, b, c) \equiv (a \rightarrow b^\prime) \oplus (b \rightarrow c^\prime) \oplus (c \rightarrow b^\prime) $
	}

	\question{4. Negate the formula and convert it to CNF} %% TODO check !!!

	\[ ((a \rightarrow b^\prime) \wedge (a \wedge \neg b) \wedge (b^\prime \wedge \neg a^\prime)) \rightarrow (b \rightarrow \neg a^\prime) \]

	\answer{%
	\begin{enumerate}		

		\item Negate the formula \\
			\[ \neg (((a \rightarrow b^\prime) \wedge (a \wedge \neg b) \wedge (b^\prime \wedge \neg a^\prime)) \rightarrow (b \rightarrow \neg a^\prime)) \] 
	  
		\item Remove $\rightarrow$ using $A \rightarrow B \equiv \neg A \vee B$ property \\
			\[ \neg (((\neg a \vee b^\prime) \wedge (\neg a \vee \neg b) \wedge (b^\prime \wedge \neg a^\prime)) \rightarrow (\neg b \vee \neg a^\prime)) \]  
			\[ \equiv \]
			\[ \neg (\neg ((\neg a \vee b^\prime) \wedge (\neg a \vee \neg b) \wedge (b^\prime \wedge \neg a^\prime)) \vee (\neg b \vee \neg a^\prime)) \] 

		\item Push $\neg$ inwards using De Morgan's Laws
			\[ ((\neg a \vee b^\prime) \wedge (\neg a \vee \neg b) \wedge (b^\prime \wedge \neg a^\prime)) \wedge \neg (\neg b \vee \neg a^\prime) \] 
			\[ \equiv \]
			\[ ((\neg a \vee b^\prime) \wedge (\neg a \vee \neg b) \wedge (b^\prime \wedge \neg a^\prime)) \wedge (b \wedge a^\prime) \] 

		\item Distribute
			\[ b \wedge ((\neg a \vee b^\prime) \wedge (\neg a \vee \neg b) \wedge (b^\prime \wedge \neg a^\prime)) \wedge a^\prime \wedge  ((\neg a \vee b^\prime) \wedge (\neg a \vee \neg b) \wedge (b^\prime \wedge \neg a^\prime)) \] 
			\[ \equiv \]
			\[ 
				\begin{split}
					b \wedge (\neg a \vee b^\prime) \wedge b \wedge (\neg a \vee \neg b) \wedge b \wedge (b^\prime \wedge \neg a^\prime) \\
					\wedge\  
					a^\prime \wedge (\neg a \vee b^\prime) \wedge a^\prime \wedge (\neg a \vee \neg b) \wedge a^\prime \wedge (b^\prime \wedge \neg a^\prime)
			 \end{split}
			\] 

			\item Simplify (remove duplicate elements because $ A \wedge A \equiv A$)

			\[ (\neg a \vee b^\prime) \wedge b \wedge (b^\prime \wedge \neg a^\prime) \wedge (\neg a \vee \neg b) \wedge a^\prime  \] 

	\end{enumerate}		

	\hline
}

\question{%
	5. Explain the principle of unit propagation from DPLL, 
	and use it to show that CNF formula of the previous question is unsatisfiable.}

	\answer{a)
	Unit propagation refers to a technique that is used to simplify
	logical formulas.
	When we have a clause that \textit{only} contains a single literal
	such as $ x $, in all other clauses that contain $ x $, we can
	replace it with $ True $.}

	\answer{b)
	We can use it to show that the previous formula is unsatisfiable by demonstrating that it is a contradiction.
	We simplify it and continue to substitute $ True $ for single literal clauses.
	When we rearrange the formula, it becomes clear that we have $ b \wedge \neg b $, this is a contradiction.

	\[ (True \vee b^\prime) \wedge True \wedge \neg b \wedge b^\prime \wedge \neg a^\prime \wedge b  \]
	\[ (True \vee True) \wedge True \wedge \neg b \wedge True \wedge \neg a^\prime \wedge b \]
	\[ \neg b \wedge True \wedge \neg a^\prime \wedge b \]
	\[ \neg b \wedge \neg a^\prime \wedge b \]
	\[ \neg b \wedge b \wedge \neg a^\prime \]
	}

\question{6. Comment on the validity of the original formula.} 

\answer{%
The formula is not satisfiable and therefore not valid as satisfiability is
a prerequisite for validity. In other words, because that logical statement
was a contradiction (\textit{assumed something can be False and True at the same time}), there was no way in which it could have been true and
because of that it is not valid.
}

\topic{First Order Logic}

	\question{%
		1. Write a formula in first-order logic, using the parent relation, 
	that states that two entities $x$ and $y$ are siblings if they share a parent}

	\answer{$ \forall{i}.\forall{j}.siblings(i, j) \equiv (parent(x, i) = parent(x, j)) $}

	\question{%
		2. Assuming there is an equality relation $ \equiv $ on $ P $ 
		such that $ x \equiv y $ means $ x $ and $ y $ are the same entity, 
	write a formula in first order logic stating that every entity has two distinct parents.}

	\answer{%
		$ \exists!{x}.P(x) \equiv \exists{x}(P(x) \wedge \forall{y}(P(y) \rightarrow y = x))  $
		\footnote{https://math.stackexchange.com/questions/228285/how-can-i-get-the-negation-of-exists-unique-existential-quantification} \\
		$ \forall{i}.\exists!{y}.\exists!{x} (parent(x, i) \wedge parent(y, i) \wedge \neg (x \equiv y)) $ 
	}

	\question{3. Prove that the following sequent is valid:}

	\[ parent(p, q).\forall{x}.\forall{y}(parent(x, y) \rightarrow \exists{z}.parent(z,x)) \vdash \exists.parent(z, p) \]

	\hline

	\begin{enumerate}		
		\item $ parent(p, q).\forall{x}.\forall{y}(parent(x, y) \rightarrow \exists{z}.parent(z,x)) $ [premise]
		\item  ---
		\item $ \exists.parent(z, p) $ [conclusion] 1 -- ?
	\end{enumerate}		

	\hline

\question{4. Interpret the following formula into simple English statement}

$ \forall{x}.\exists{y}.parent(y, x) \rightarrow \forall{a}.\exists{b}.parent(a, b) $

\answer{If all children have a parent, then all parents have a child.}

\question{5. The sequent below is not valid. Show this by proving a 
	counter-example: a concrete definition of the universe $ P $ (a set) 
	and the relation parent. Explain why this is a counterexample.}

\[ 
	\vdash \forall{x}.\exists{y}.parent(y, x) \rightarrow \forall{a}.\exists{b}.parent(a, b) 
\]

\begin{flushleft}
	\textbf{In English}\\
	if all children have a parent then all parents have a child.
\end{flushleft}

\begin{enumerate}		
	\item $ \forall{x}.\exists{y}.parent(y, x) \rightarrow \forall{a}.\exists{b}.parent(a, b) $ [premise] \\ 
  \hline
	\item  $ \let{P} = \{\ x\ |\ x\ \} $  \\
  \hline
	\item $ False $ [conclusion]
\end{enumerate}		


\question{6. What is the smallest possible counterexample universe $ P $ and relation parent. Explain your reasoning.}

\question{7. Prove that the following sequent is valid.}

\[
	\forall{x}.\forall{y}.parent(x, y) \rightarrow \neg(x \equiv y) \vdash \neg \exists{x}.parent(x, x)
\]

ie. if someone is a parent and has a child, then that child cannot be it's parent.

\begin{enumerate}		
	\item $ \forall{x}.\forall{y}.parent(x, y) \rightarrow \neg(x \equiv y) $ [premise] \\
	\hline
	\item $ parent(x_i, y_j) \rightarrow \neg (x_i \equiv y_j)  $ $ \forall_e $ 1
	\item $ \neg parent(x_i, y_j) \vee \neg (x_i \equiv y_j) $  because $ P \rightarrow Q \equiv \neg P \vee Q $
	\item $ \neg (parent(x_i, y_j) \wedge (x_i \equiv y_j)) $  De Morgan's Law \\
	\hline
	\item $ \neg \exists{x}.parent(x, x) [conclusion] $
\end{enumerate}		

\topic{Regular Lanugages and Finite Automata}

\question{1. For the reular expression $ (A|AB|ABB)^* $ do the following steps (and justify them):}

\begin{itemize}		
	\item[a)] translate it into an NFA
		\answer{\centeredimg{rl_1_a}}

	\item[b)] remove (semantic preserving) the $ \epsilon $ transitions
		\answer{\centeredimg{rl_1_b}}

	\item[c)] create the corresponding DFA
		\answer{\centeredimg{rl_1_c}}
\end{itemize}		

\pagebreak

\question{%
	2. Consider the following language: \\ A word is in the language if and
	only if it contains an even number of $A$s an odd number of $B$s and
	precisely one $C$.
}
	\begin{enumerate}		
		\item[a)] Design a DFA for this language
			\answer{%
				Not possible.
			}
		\item[b)] Explain your design  (what information do you "store " in your states?)
	\end{enumerate}		

\pagebreak

\section*{Additional Sources}

\begin{itemize}[noitemsep]		
  \item https://en.wikipedia.org/wiki/Negation
  \item https://en.wikipedia.org/wiki/Distributive\_property
  \item https://en.wikipedia.org/wiki/Conjunctive\_normal\_form
  \item https://en.wikipedia.org/wiki/De\_Morgan\%27s\_laws
  \item https://en.wikipedia.org/wiki/Modus\_ponens
  \item https://en.wikipedia.org/wiki/Exclusive\_or
  \item https://en.wikipedia.org/wiki/Finite-state\_machine
  \item https://en.wikipedia.org/wiki/Uniqueness\_quantification
\end{itemize}		

\end{document}

	%% vim: foldmethod=marker conceallevel=0:
